% $Id: title.tex,v 1.24 2016/03/30 17:18:10 jfnielse Exp $

%\title{toppe.e: A general-purpose pulse sequence for GE MRI scanners}
%\date{March 9, 2016}
%\author{Jon-Fredrik Nielsen, Ph.D. \\ fMRI Laboratory, University of Michigan \\ {\tt jfnielse@umich.edu} } 
%\maketitle

\begin{titlepage}
~\\[0.5in]
\begin{centering}
\rule{\textwidth}{5pt}
~\\[0.5in]
{\bf \huge The TOPPE pulse programming environment} \\ [0.4in]
{\bf \huge for GE MRI scanners} \\ [0.6in]
{\tt This document applies to version~\toppeversion~of the pulse sequence (\toppe).   } \\ [0.1in]
{\tt Tested on a GE Discovery MR750 scanner running software version DV25.1\_R01.   } \\ [0.6in]
{\tt Version of this document:~\toppeversion-\today} \\ [1in]
{\large Jon-Fredrik Nielsen, Ph.D.} \\ [0.1in]
{\tt jfnielse@umich.edu} \\ [1in]
This pdf is available from the TOPPE website: \toppeweb. \\ [0.25in]
This document is open-source. Latex code can be found in the UserGuide folder in the following repository: \url{https://github.com/toppeMRI/toppemri.github.io} \\
Alternatively, you can get a copy of the repository by typing the following in a Linux console:\\
{\tt git clone https://github.com/toppemri/toppemri.github.io}. \\ [0.1in]
%fMRI Laboratory, University of Michigan \\
% {\includegraphics[width=0.7\textwidth]{logo}}
% \vfill
%{\small This pulse sequence programming framework is licensed under the Creative Commons Attribution 4.0 International License. To view a copy of this license, visit \url{http://creativecommons.org/licenses/by/4.0/ }}\\
\rule{\textwidth}{5pt}
\end{centering}
\end{titlepage}


